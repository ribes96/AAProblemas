\section*{Problem statement}
\subsection*{The problem}

We have to prove that board games like chess or go in $nxn$ boards are in EXP.

\subsection*{The solution}

This is easy. We only have to give an exponential algorithm that solves such a 
game. 

The algorithm is as follows. We construct a tree of possible states
that can be produced from the initial states by moving pieces around. 
For every configuration in the board, there are (at most) two states/nodes in the
tree. One for the case that it is white's turn to play, and one for the 
case that it is black's turn. 

Let us consider that in any game there are c kinds of pieces. For example, in chess we have $c=6$ (king,queen,rook,bishop,knight,pawn). That means that in 
every square of the board can have none or one of 2*c pieces (every piece can 
be either black or white).  Therefore, the number of possible 
configurations on a $n*n$ board is $(2*c+1)^{n^2}$. 

Clearly, this is a directed tree with $(2*c+1)^{n^2}$ nodes. Also, there are back edges, although these will not trouble us a lot, as shown later. While 
creating this tree, we find some states in which either black or white 
wins. These are the leaves of the tree. 

The creation of this tree takes exponential time. Why? Let us describe the 
constuction process with in detail.

Each state is in the form $s_i= <board_i, white\_plays_i>$. The first variable gives us the board configuration and 
the second is a boolean that is true if it is white's turn to play and false otherwise. 

We begin with an initial state $s_0$, given as input to the problem. Then, we start exploring in a BFS manner. Firstly, we 
create the $p(n)$ states that can be generated by this initial state, where $p(n)$ is a polynomial of size n. We add the newly
created states to a queue, and then we repeat the process for each one of the states in the queue, but without creating 
duplicate states. In this manner, we shall create no more than $2(2c+1)^{n^2}$ nodes, and since for each node we 
try p(n) moves, the creation of tree is complete after $O((2c+1)^{n^2}p(n))$ steps.

This tree will help us check if one of the two players has a winning strategy.

This concludes the proof.
