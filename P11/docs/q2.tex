<<<<<<< HEAD
\textbf{La respuesta a Q2 empieza aquí} 
\\
=======
\textbf{La respuesta a Q2 empieza aquí}
>>>>>>> f87786cdac186b5648f9d7bc8ee9d7750a66e22b
En los casos en los que $j \notin S_i$, $\tilde{y}_{ij} = 0$, y eso siempre es menor o igual.
Para el resto de casos, podemos sustiuir $\tilde{y}_{ij}$ por el valor indicado y la demostración se transforma de la siguiente manera:
\begin{equation}
 \frac{y^*_{ij}}{\Sigma_{j \in S_i} y^*_{ij}} \leq 2y^*_{ij}
\end{equation}

\begin{equation}
 \frac{y^*_{ij}}{2y^*_{ij}} \leq \Sigma_{j \in S_i} y^*_{ij}
\end{equation}

\begin{equation}
 0.5 \leq \Sigma_{j \in S_i} y^*_{ij}
\end{equation}

\begin{equation}
 0.5 > \Sigma_{j \notin S_i} y^*_{ij}
\end{equation}

El ultimo paso es valido porque $\Sigma_{j} y^*_{ij} = 1$. Ahora, vamos a demonstrar que la última relación es cierta. Primero, es verdad que:

\begin{equation}
 \Sigma_{j \notin S_i} c(i,j) y^*_{ij} \leq \Sigma_{j \in M} c(i,j) y^*_{ij} = \tilde{c_i}
\end{equation}
\\
Pero también tenemos que $\Sigma_{j \notin S_i} c(i,j) y^*_{ij} > \Sigma_{j \notin S_i} 2 \tilde{c_i} y^*_{ij}$ . Entonces:
\\
\begin{equation}
 2* \tilde{c_i} * \Sigma_{j \notin S_i} y^*_{ij}  < \tilde{c_i}
\end{equation}

\begin{equation}
 \Sigma_{j \notin S_i} y^*_{ij} < \frac{1}{2}
\end{equation}

\textbf{La respuesta a Q2 termina aquí}

