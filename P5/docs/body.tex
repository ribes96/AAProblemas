\section*{Statement}

We are feeling experimental and want to create a new dish. There are various ingredients we can choose from and we'd like to use as many of them as possible, but some ingredients don't go well with others. If there are $n$ possible ingredients (numbered 1 to $n$, we write down the $n\times n$ matrix D giving the \textit{discord} between any pair of ingredients. This \textit{discord} is a real number between $0.0$ and $1.0$, where $0.0$ means ``they go together perfectly'' and $1.0$ means ``they don't go together''. For example, if $D[2,3] = 0.1$ and $D[1,5] = 1.0$, then ingredients 2 and 3 go together pretty well whereas 1  and 5 clash badly.

Notice that D is necessarely symmetric and that the diagonal entries are always $0.0$. Any set of ingredients always incurs a penalty which is the sum of all discors values between pairs of ingredients. For instance the set of ingredients $\{1,2,3\}$ incurs a penalty of $D[1,2] + D[1,3] + D[2,3]$. We want the penalty to be small.

\paragraph{EXPERIMENTAL CUISINE}
Given n ingredients, and the discord $n \times n$ matrix D and some number $p$, compute the maximum number of ingredients we can choose with penaly $\leq p$
\\ \\
Show that if \ExC is solvable in polinomial time, then is so 3SAT.

\section*{Answer}
One way to prove this is by reducing \textsc{3sat} to \ExC. Once this is done, we can safely assume that if \ExC is solvable in polynomial time, so is \textsc{3sat}.
\\ \\
We will proceed with this reduction as follows:
\begin{enumerate}
  \item By reducing \textsc{3sat} to \textsc{Independent Set}
  \item By reducing \textsc{Independent Set} to \ExC
\end{enumerate}
The reduction from \textsc{3sat} to \textsc{Independent Set} is proven here\cite{indep_set_proof}.

Now we need to reduce \textsc{Independent Set} to \ExC.

Given an instance of \textsc{Independent Set}, $G$, where $G$ is a graph, construct the following:

Set $D$ to be an incidence matrix of $G$, where $a_{ij}=0$ if the graph's vertices do not share a common edge, and $a_{ij}=1$ if they are, and assign to the maximum discordance $d$ the value of 0.

If we feed this values to an algoritm that solves \ExC, the result that it yields is a subset of verticies $V$ such that $V$ is an \textsc{Independent Set} of $G$.

\paragraph{Proving that this reduction is correct}
Given a graph $G = (V,E)$ and a vertex set $I \subseteq V$ that is the independent set of $G$ with highest cardinality, it is easy to see that, when we transform this configuration using the steps described above, it is a valid solution to \ExC, and adding one more vertex to is not. This means it is the maximum subset that satisfies the problem.
