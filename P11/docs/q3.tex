\textbf{La respuesta a Q3 empieza aquí}

To say this approximation is feasible is equilvalent to say that


\begin{equation}\label{eq:pythagoras}

  \tilde{x}_{j} \lequal 2x_{i} or \tilde{x}_{j} = 1
  
\end{equation}
  
  2. y~_ij <= 2y*_ij (By Q2)


Applying transformations to 
  1-yij + xj >= 1

We have:
  -y_ij + x_j >= 0
  x_j >= y_ij

That means that the formula implies that for a feasible solution,
for all ij this holds true. Specifically, for the optimal solution:
  x*_j >= y*_ij

Also:
  2x*_j >= 2y*_ij
  
So:
  y~_ij <= 2y*_ij <= 2x*_j <= x~_j

And, by transitivity:
  y~_ij <= x~_j

The other proof is direct. We only have to observe that
  y~_ij <= 2y*_ij

Then, for each ij holds true that
  c_ij * y~_ij <= c_ij * 2y*_ij

The other restriction
  yi1 + yi2 + ... + yim >= 1;

Can be easily proven to be true.
Let us see that in fact this restriction
  y~_ij = y*_ij/summ(j in S_i) of y*_ij

Is dividing an element of a set between the sum of all the elements of a set
and it is doing it so for every element in the set. This makes it so the sum
is always going to give 1. This makes the previous restriction true.

\textbf{La respuesta a Q3 acaba aquí}
