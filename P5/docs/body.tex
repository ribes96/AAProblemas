\section*{Problem statement}
\subsection*{The \exc{} problem}
We are feeling experimental and want to create a new dish. There are various ingredients we can choose from and we'd like to use as many of them as possible, but some ingredients don't go well with others. If there are $n$ possible ingredients (numbered 1 to $n$, we write down the $n\times n$ matrix D giving the \textit{discord} between any pair of ingredients. This \textit{discord} is a real number between $0.0$ and $1.0$, where $0.0$ means ``they go together perfectly'' and $1.0$ means ``they don't go together''. For example, if $D[2,3] = 0.1$ and $D[1,5] = 1.0$, then ingredients 2 and 3 go together pretty well whereas 1  and 5 clash badly.

Notice that D is necessarely symmetric and that the diagonal entries are always $0.0$. Any set of ingredients always incurs a penalty which is the sum of all discors values between pairs of ingredients. For instance the set of ingredients $\{1,2,3\}$ incurs a penalty of $D[1,2] + D[1,3] + D[2,3]$. We want the penalty to be small.

\subsection*{Exercise}
Given n ingredients, and the discord $n \times n$ matrix D and some number $p$, compute the maximum number of ingredients we can choose with penaly $\leq p$

Show that if \exc{} is solvable in polinomial time, then is so 3SAT.

\section*{Answer}
A way to prove that if \exc{} is solvable in polinomial time, then so is \tsat{} is by reducing in polynomial time \tsat{} to \exc{}. If a reduction was found, we'd have:
\\ \\
\tsat{} $\karp{}$ \exc{}
\\ \\
Which means that \exc{} is harder or as hard to solve as \tsat{}. Once we've proven this to be true, there is no doubt that if we assume that \exc{} is solvable in polynomial time, given that \tsat{} $\karp{}$ \exc{}, we have that \tsat{} must be solvable in polynomial time, because it is easier or as easy to solve as \exc{}, given that we find a reduction.

\\ \\
We will proceed with this reduction as follows:
\begin{enumerate}
  \item By reducing \tsat{} to \is{}
  \item By reducing \is{} to \exc
\end{enumerate}
The reduction from \tsat{} to \is{} is proven here\cite{indep_set_proof}.

Now we need to reduce \is{} to \exc{}.

Given an instance of \is{}, $G$, where $G$ is a graph, construct the following:

Set $D$ to be an incidence matrix of $G$, where $a_{ij}=0$ if the graph's vertices do not share a common edge, and $a_{ij}=1$ if they are, and assign to the maximum discordance $d$ the value of 0.

If we feed this values to an algoritm that solves \exc{}, the result that it yields is a subset of verticies $V$ such that $V$ is an \is{} of $G$.

\paragraph{Proving that this reduction is correct}
Given a graph $G = (V,E)$ and a vertex set $I \subseteq V$ that is the independent set of $G$ with highest cardinality, it is easy to see that, when we transform this configuration using the steps described above, it is a valid solution to \exc{}, and adding one more vertex to is not. This means it is the maximum subset that satisfies the problem.
