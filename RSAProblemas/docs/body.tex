\section*{Exercise 1}
The idea is to use Euler's theorem (generalization of Fermat's Little Theorem).
\[ a^{\phi(n)} = 1 mod n \] where a is coprime to n. 

This helps us compute powers modulo n. E.g, in the exercise $n=10, \phi(n) =4, a=3$. It is true that $28=4*7$, therefore we can be sure that $3^{28}=1 (mod10)$
\\

The second case is different: $n=15, \phi(n)=8, a=3$. Here we notice that lcd(3,15)=3, and therefore we cannot use Euler's Theorem.

However, we happily discover that $3^4 mod 15 = 3$. That means that we have actually found a repetition pattern in the chain 
$3^1,3^2,3^3,3^4\ldots $ and this means that $3^{3k+1}=3$. But we know that $200= 3*66+2$, therefore : $3^{200}=3^{3*66+1}*3=3^2=9$.

\section*{Exercise 2}

We first compute $x=b^c mod \phi(p)$ \\
Then we compute  $a^x mod p$. 

Why is this correct? Simply because for every p,a, as Euler tells us that$a^{\phi(p)}=1 mod p$, we have the following $ a^k = a^{k+\phi(p)*l} mod p $.


\section*{Exercise 3}
Solution.
\section*{Exercise 4}
SOLUTION.
\section*{Exercise 5}
Solution.


