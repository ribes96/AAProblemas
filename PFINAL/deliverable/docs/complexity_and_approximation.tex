\section{A $(log^2(n))$ Approximation}

There exists an algorithm that runs in subexponential time and achieves an
$(log^2(n))$ approximation of the min.cut linear arrangement problem. 

Source: \cite{vazi01}

\textbf{TODO: examine explain how the time is subexponential}
\textbf{TODO: explain the proof below in a better way (look at Vazirani-Approximation algorithms)}

Problem 21.27 (Minimum b-balanced cut)	

% %Given an undirected graph
% G = (V, E) with nonnegative edge costs and a rational b, 0 < b =< 1/2, find a
% minimum capacity cut (S, S) such that b * n <= |S| < (1 − b) * n.
% A b-balanced cut for b = 1/2 is called a bisection cut, and the problem of
% finding a minimum capacity such cut is called the minimum bisection problem.
% We will use Theorem 21.24 to obtain a pseudo-approximation algorithm for
% Problem 21.27 – we will find a (1/3)-balanced cut whose capacity is within
% an O(log n) factor of the capacity of a minimum bisection cut (see the notes
% in Section 21.8 for a true approximation algorithm).
% For V ⊂ V , let G V denote the subgraph of G induced by V . The
% algorithm is: Initialize U ← ∅ and V ← V . Until |U | ≥ n/3, find a minimum
% expansion set in G V , say W , then set U ← U ∪W and V ← V −W . Finally,
% let S ← U , and output the cut (S, V − S).
% 
% Problem 21.29 (Minimum cut linear arrangement) Given an undi-
% rected graph G = (V, E) with nonnegative edge costs, for a numbering of its
% vertices from 1 to n, define S i to be the set of vertices numbered at most i,
% for 1 <= i <= n − 1; this defines n − 1 cuts. The problem is to find a numbering
% that minimizes the capacity of the largest of these n−1 cuts, i.e., it minimizes
% max{c(S i )| 1 <= i <= (n − 1)}.
% Using the pseudo-approximation algorithm obtained above for the (1/3)-
% balanced cut problem, we will obtain a true O(log 2 n) factor approximation
% algorithm for this problem. A key observation is that in any arrangement,
% S n/2 is a bisection cut, and thus the capacity of a minimum bisection cut
% in G, say β, is a lower bound on the optimal arrangement. The reason we
% get a true approximation algorithm is that the (1/3)-balanced cut algorithm
% compares the cut found to β.
% The algorithm is recursive: find a (1/3)-balanced cut in G V , say (S, S),
% and recursively find a numbering of S in G S using numbers from 1 to |S|
% and a numbering of S in G S using numbers from |S| + 1 to n. Of course,
% the recursion ends when the set is a singleton, in which case the prescribed
% number is assigned to this vertex.
% Claim 21.30 The algorithm given above achieves an O(log 2 n) factor for the
% minimum cut linear arrangement problem.
